\section{Singular Values Decomposition}
提前介绍后面要用到的性质。

[矩阵$\bm{A^TA}$和$\bm{AA^T}$的性质及证明]\url{https://blog.csdn.net/Europe233/article/details/86720864}

性质:$\bm{A^TA}$和$\bm{AA^T}$具有相同的非零特征值

证明1:\url{https://blog.csdn.net/Europe233/article/details/86727078}

证明2:
\begin{enumerate}
    \item 假设$\bm{x} \neq \bm{0}$,$\bm{A}^T\bm{Ax}=\lambda \bm{x}$,等式左右两边同时左乘一个$\bm{A}$
    得$\bm{\rightarrow} (\bm{AA}^T)\bm{Ax}=\lambda \bm{Ax}$;
    \item 同理,假设$\bm{x} \neq \bm{0}$,$\bm{AA}^T\bm{x}=\lambda \bm{x}$,左右两边同时左乘一个$\bm{A^T}$,
    得$\bm{\rightarrow} (\bm{A}^T\bm{A})\bm{A}^T\bm{x}=\lambda \bm{A^T}\bm{x}$。
    \item 经检验,维度是匹配的。这说明在$\lambda \neq 0$时上述式子成立,也即$\bm{A^TA}$和$\bm{AA^T}$具有相同的非零特征值。
\end{enumerate}

\subsection{SVD的目的}
从空间变换的角度出发
\begin{itemize}
    \item 对称阵的谱分解(特征值分解):通过矩阵$\bm{A}$将一组基映射到同一组基。
    \item SVD的目的是:通过任意矩阵$\bm{A}$将一组单位正交基映射到另一组单位正交基。
\end{itemize}

SVD的两种表示(下面的向量和矩阵都没有加粗,自行辨识):
$$A v_{i}=\sigma_{i} u_{i}$$
$$\begin{array}{c}
(m \text { by } n)(n \text { by } r) \\
A V_{r}=U_{r} \Sigma_{r} \\
(m \text { by } r)(r \text { by } r)
\end{array}
\\
A\Bigg[\begin{array}{lll}
v_{1} & \cdots & v_{r}
\end{array}\Bigg]=\Bigg[\begin{array}{lll}
u_{1} & \cdots & u_{r}
\end{array}\Bigg]\left[\begin{array}{cccc}
\sigma_{1} & & & \\
& \cdot & \\
& & \cdot & \\
& & & \sigma_{r}
\end{array}\right]$$


书上7.2的直接给出向量$u,v$是从四个基本子空间的正交单位基而来。

这里给出个人想法:
\begin{itemize}
    \item 直接从目标的表达式$\bm{A V=U \Sigma}$(此处$\bm{V,U,\Sigma}$为方阵)看出,
    每个向量$\bm{u}_i$是矩阵$\bm{A}$的列向量的线性组合,所以$\bm{u}_i$处于$\bm{A}$的列空间中!
    \item 倘若按照SVD的目的,从一组单位正交基映射到另一组单位正交基,那么矩阵$\bm{U,V}$便都是正交矩阵,
    所以对目标表达式左乘一个$U^T$,右乘一个$V^T$然后再转置得到:$\bm{A}^T\bm{U} = \bm{V\Sigma}$。
    由此便可以按列向量展开,然后看出每个$\bm{v}_i$向量都是矩阵$A$的行向量的线性组合,即$\bm{v}_i$处于
    $\bm{A}$的行空间中。
    \item 所以书上一开始就直接去$\bm{A}$的列空间和行空间里面去找规范正交基
    \item 后面分别找零空间和左零空间的规范正交基是因为要找满足分别与先前两组规范正交基正交的另外两组基,
    并且恰好分别能将其扩展为方阵!
\end{itemize}

\url{https://www.zhihu.com/question/22237507}、\url{https://blog.csdn.net/zhongkejingwang/article/details/43053513},
的推导过程:
\\
\textbf{假设}找到了一组正交的基$\left\{v_{1}, v_{2}, \ldots, v_{n}\right\}$,
通过矩阵A将这组正交基映射为$\left\{A v_{1}, A v_{2}, \ldots, A v_{n}\right\}$,令映射后的向量两两正交即:
$\left(\mathrm{A} v_{i}\right)^{T} A v_{j}=v_{i}^{T} A^{T} \mathrm{A} v_{j}=0$。
根据假设,存在$v_{i}^{T} v_{j}=0$,由此看出,如果\textbf{选用矩阵$A^TA$这个实对称阵的特征向量作为$v_i$的话,就满足要求},
也即$v_{i}^{T} A^{T} \mathrm{A} v_{j}= v_{i}^{T} \lambda_{j} v_{j} =\lambda_{j} v_{i}^{T} v_{j} =\lambda_{j} v_{i}, v_{j}=0$。
所以这里便找到了另一组对应的正交基$\left\{A v_{1}, A v_{2}, \ldots, A v_{n}\right\}$!

然后将映射后的正交基单位化:
\begin{itemize}
    \item 因为$(A v_{i})^T A v_{i}=v_{i}^T (A^TA)v_i = v_{i}^T \lambda_{i} v_{i}=\lambda_{i}$($v_i$是正交单位向量)
    \item 所以有$\left||A v_{i}\right||^{2}=\lambda_{i} \geq 0$
    \item 取单位向量:$u_{i}=\dfrac{A v_{i}}{\left||A v_{i}\right||}=\dfrac{1}{\sqrt{\lambda_{i}}} A v_{i}$
\end{itemize}
由此可得,$A v_{i}=\sigma_{i} u_{i}, \sigma_{i}($奇异值$)=\sqrt{\lambda_{i}}, 0 \leq i \leq \mathrm{k}, \mathrm{k}=\operatorname{Rank}(\mathrm{A})$

\textbf{注意}:对$u_{i}=\dfrac{A v_{i}}{\left||A v_{i}\right||}=\dfrac{1}{\sqrt{\lambda_{i}}} A v_{i}$左右两边同时左乘上$AA^T$
    得
\begin{itemize}
    \item $AA^Tu_{i}=\dfrac{1}{\sqrt{\lambda_{i}}} AA^TA v_{i}$
    \item 即$AA^Tu_{i}=\dfrac{1}{\sqrt{\lambda_{i}}} A(A^TA) v_{i}=\dfrac{1}{\sqrt{\lambda_{i}}} A \lambda_i v_{i}$
    \item 即$AA^Tu_{i}=\lambda_i \dfrac{1}{\sqrt{\lambda_{i}}} A  v_{i}$
    \item 即$AA^Tu_{i}=\lambda_i u_i$
\end{itemize}
\textbf{所以,向量$u_i$是矩阵$AA^T$的特征向量,且特征值相同(对应最开始$AA^T, A^TA$的性质)}

将SVD用向量写出来:$A=U \Sigma V^{\mathrm{T}}=u_{1} \sigma_{1} v_{1}^{\mathrm{T}}+\cdots+u_{r} \sigma_{r} v_{r}^{\mathrm{T}}$,
这便是$\bm{A}$的满秩分解,将$\bm{A}$分解为只有$rank(\bm{A})$个矩阵之和!当然$\bm{U,V}$ 扩展成方阵也是,
新增进去的向量不起作用,以及$\bm{\Sigma}$多出来的部分里面值为0。

\subsection{瑞利熵(Rayleigh quotient)}
实数上的定义:
$$r(\boldsymbol{x})=\dfrac{\boldsymbol{x}^{\mathrm{T}} S \boldsymbol{x}}{\boldsymbol{x}^{\mathrm{T}} \boldsymbol{x}}$$
可以参考\url{https://www.cnblogs.com/xingshansi/p/6702188.html?utm_source=itdadao&utm_medium=referral}A-普通瑞利熵
部分,得出对该式求得最大值时的条件就是$S x=r(x) x$,也就是说当$r(\bm{x})$为$S$的特征值,$\bm{x}$为$S$的特征向量时满足要求!
与书上的表述一致。

