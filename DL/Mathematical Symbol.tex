\documentclass[UTF8]{article}
\usepackage{ctex}
\usepackage{amsmath}
\usepackage{graphicx}
\usepackage{bm}
\usepackage[colorlinks, linkcolor=black]{hyperref}


\title{
    \begin{center}{\Huge \textit{Notes: Neural Networks}}
    %\\{{\itshape This a subtitle}}
    \end{center}}

\begin{document}

    \maketitle
    \tableofcontents

    \newpage


    \section{感知机 Perceptron}
    \textbf{前提:数据是线性可分的!}\\
    点到平面的距离公式:$d=\frac{\left|A x_{0}+B y_{0}+C z_{0}+D\right|}{\sqrt{A^{2}+B^{2}+C^{2}}}$
    \\
    输入:m个样本即$\left(x_{1}^{(0)}, x_{2}^{(0)}, \ldots x_{n}^{(0)}, y_{0}\right),\left(x_{1}^{(1)}, x_{2}^{(1)}, \ldots x_{n}^{(1)}, y_{1}\right), \ldots\left(x_{1}^{(m)}, x_{2}^{(m)}, \ldots x_{n}^{(m)}, y_{m}\right)$,
    标签y是二元类别。
    \\
    目标:找到一个超平面(hyperplane)即$\theta_{0}+\theta_{1} x_{1}+\ldots+\theta_{n} x_{n}=0$,让其中一种类别的样本都满足$\theta_{0}+\theta_{1} x_{1}+\ldots+\theta_{n} x_{n}>0$;
    而另一种类别的样本都满足$\theta_{0}+\theta_{1} x_{1}+\ldots+\theta_{n} x_{n}<0$,从而线性可分!\textbf{简化写法:}增加一个特征$x_0=1$,所以有超平面$\sum_{i=0}^{n} \theta_{i} x_{i}=0$
    \textbf{向量表示:}$$\theta_{(n+1) \times 1} \cdot x_{1 \times (n+1)}= 0$$
    \\
    感知机模型定义: $$y=\operatorname{sign}(\theta \bullet x)$$
    $$\operatorname{sign}(x)=\left\{\begin{array}{ll}{-1} & {x<0} \\ {1} & {x \geq 0}\end{array}\right.$$
    \\
    Gram矩阵(可以看成没有减去均值的协方差矩阵), 感知机\textbf{原始形式以及对偶形式}

    \section{DNN反向传播}
    \subsection{要解决的问题}
    在监督学习中往往要利用一个loss function来度量当前模型从输入得到的结果与真实值的误差。通过最小化这个误差,把整个模型往最好的方向调整即得到最好的weight和b;
    而最小化的方法便有梯度下降,牛顿法,拟牛顿法等...



\end{document}