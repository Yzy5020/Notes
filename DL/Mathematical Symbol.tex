\documentclass[UTF8]{article}
\usepackage{ctex}
\usepackage{amsmath}
\usepackage{amssymb}
\usepackage{graphicx}
\usepackage{bm}
\usepackage[colorlinks, linkcolor=black]{hyperref}


\title{
    \begin{center}{\Huge \textit{Notes: Neural Networks}}
    %\\{{\itshape This a subtitle}}
    \end{center}}

\begin{document}

    \maketitle
    \tableofcontents

    \newpage


    \section{感知机 Perceptron}
    \textbf{前提:数据是线性可分的!}\\
    点到平面的距离公式:$d=\frac{\left|A x_{0}+B y_{0}+C z_{0}+D\right|}{\sqrt{A^{2}+B^{2}+C^{2}}}$
    \\
    输入:m个样本即$\left(x_{1}^{(0)}, x_{2}^{(0)}, \ldots x_{n}^{(0)}, y_{0}\right),\left(x_{1}^{(1)}, x_{2}^{(1)}, \ldots x_{n}^{(1)}, y_{1}\right), \ldots\left(x_{1}^{(m)}, x_{2}^{(m)}, \ldots x_{n}^{(m)}, y_{m}\right)$,
    标签y是二元类别。
    \\
    目标:找到一个超平面(hyperplane)即$\theta_{0}+\theta_{1} x_{1}+\ldots+\theta_{n} x_{n}=0$,让其中一种类别的样本都满足$\theta_{0}+\theta_{1} x_{1}+\ldots+\theta_{n} x_{n}>0$;
    而另一种类别的样本都满足$\theta_{0}+\theta_{1} x_{1}+\ldots+\theta_{n} x_{n}<0$,从而线性可分!\textbf{简化写法:}增加一个特征$x_0=1$,所以有超平面$\sum_{i=0}^{n} \theta_{i} x_{i}=0$
    \textbf{向量表示:}$$\theta_{(n+1) \times 1} \cdot x_{1 \times (n+1)}= 0$$
    \\
    感知机模型定义: $$y=\operatorname{sign}(\theta \bullet x)$$
    $$\operatorname{sign}(x)=\left\{\begin{array}{ll}{-1} & {x<0} \\ {1} & {x \geq 0}\end{array}\right.$$
    \\
    Gram矩阵(可以看成没有减去均值的协方差矩阵), 感知机\textbf{原始形式以及对偶形式}

    \section{DNN反向传播}
    \subsection{要解决的问题}
    在监督学习中往往要利用一个loss function来度量当前模型从输入得到的结果与真实值的误差。通过最小化这个误差,把整个模型往最好的方向调整即得到最好的weight和b;
    而最小化的方法便有梯度下降,牛顿法,拟牛顿法等...

    \section{RNN BPTT(back propagation through time)}
    模型定义参考\href{https://www.cnblogs.com/pinard/p/6509630.html}{刘建平RNN推导}

    \subsection{前向传播}
    \begin{enumerate}
        \item 对于任意一个序列索引号t:$h^{(t)}=\sigma\left(z^{(t)}\right)=\sigma\left(U x^{(t)}+W h^{(t-1)}+b\right)$,这里的$\sigma$是tanh函数
        \item 序列索引号t时模型的输出$o^{t}$的表达式为:$o^{(t)}=V h^{(t)}+c$,$c,b$一样,都是偏置向量
        \item 最终在序列索引号t时预测输出$\hat{y}^{(t)}=\sigma\left(o^{(t)}\right)$,这里的$\sigma$是softmax函数
    \end{enumerate}

    \subsection{反向传播}
    通过梯度下降法的一轮迭代,得到模型参数$U,W,V,b,c$,\textbf{这些参数在序列的各个位置是共享的},BP时更新的是相同的参数!

    由于在序列的每个位置都有损失函数$L^{(t)}=-{y^{(t)}}^T log{\hat{y}^{(t)}}$,因此最终的损失$L$为:$L=\sum_{t=1}^{\tau} L^{(t)}$

    下面的推导详见草稿纸笔记。。。

    \section{Convolution}

    \href{https://en.wikipedia.org/wiki/Convolution}{Convolution (from en.wikipedia)}
    、
    \href{https://zh.wikipedia.org/wiki/%E5%8D%B7%E7%A7%AF}{卷积 (from zh.wikipedia)}

    卷积的连续、离散定义见上述两个链接。

\hspace*{\fill} \\
    \href{https://zhuanlan.zhihu.com/p/150737244}{卷积性质及其证明}:
    \begin{itemize}
        \item 第2条微分性质是连续卷积的;
        \item 第5条是从信号处理的角度推导出离散的卷积形式;
    \end{itemize}

    在上述两个维基百科链接中,\textbf{卷积的微分性质同时适用于连续和离散的形式,只是在离散情况下微分算子D
    变成差分算子!}

\hspace*{\fill} \\
    \href{https://www.zhihu.com/question/22298352/answer/637156871}{卷积的含义及直观地理解}

\hspace*{\fill} \\
    复旦邱锡鹏《神经网络与深度学习》(下文均称作nndl)第5章 5.1详细介绍了数学上的一维、二维卷积定义以及互相关定义!
    (\href{https://www.zybuluo.com/hanbingtao/note/485480}{零基础入门深度学习(4) - 卷积神经网络}、
    \href{https://www.cnblogs.com/pinard/p/6483207.html}{卷积神经网络(CNN)刘建平}
    与上述书本介绍的定义是类似的,只是书上的卷积核下标从1开始,而这两个链接卷积核的下标从0开始!)

    从信号叠加的角度来看比较容易理解,\textbf{一维卷积}$y_{t}=\sum_{k=1}^{K} w_{k} x_{t-k+1}$表示
    在时刻$t$收到的信号$y_t$为当前时刻产生的信息$\{x_{t}\}$和以前时刻延迟信息$\{x_{t-k+1}\}$的叠加。
    一般来说,每个时间点$t$都有完整时序(数目)变化$[1,K]$的$w_{k}$被称作卷积核或滤波器(在二维(图像)情况下同理)。

    \textbf{二维卷积}:给定一个图像$\boldsymbol{X} \in \mathbb{R}^{M \times N}$和一个滤波器$\boldsymbol{W} \in \mathbb{R}^{U \times V}$
    一般$U \ll M, V \ll N$,则$y_{i j}=\sum_{u=1}^{U} \sum_{v=1}^{V} w_{u v} x_{i-u+1, j-v+1}$,
    这里假设输出$y_{ij}$的下标$(i, j)$从$(U, V)$开始($Y$矩阵的第一个元素),这样输出矩阵才不会出现下标为负的情况。
    可以看出在二维卷积的定义下,\textbf{卷积核是翻转了180度}在输入矩阵上滑动并做加权求和操作。参考零基础入门链接里的
    二维卷积定义的图示并将其写成表达式即可理解!
    (翻转指从两个维度(从上到下、 从左到右)颠倒次序, 即旋转180度)

    \textbf{互相关}:设定如上述二维卷积,但公式变为$y_{i j}=\sum_{u=1}^{U} \sum_{v=1}^{V} w_{u v} x_{i+u-1, j+v-1}$
    和二维卷积公式对比可知, \textbf{互相关的卷积核不进行翻转}。

    \subsection{Back Propagation}
    书本nndl 以及上述两个链接均采用链式法则+逐标量的推导方式来找到反向传播的矩阵表达式;
    但是直接用矩阵、向量的链式法则是推导不出来的(因为是互相关操作而不是矩阵乘法)!

    nndl中 Sec 5.1.4.2从互相关定义出发,逐分量给出了标量对进行了互相关操作后的矩阵的导数!
    这与上述两个链接里求反向传播参数中使用到的推导方法是一样的!

    理论推导看上述两个链接+nndl第五章即可!

    \subsubsection{Naive实现}

    具体细节说明见\href{https://www.zybuluo.com/hanbingtao/note/485480}{零基础入门深度学习(4)-卷积神经网络};
    实现参考:\href{https://github.com/hanbt/learn_dl/blob/master/cnn.py}{零基础入门深度学习(4)-卷积神经网络 具体实现};
    \href{https://github.com/ddbourgin/numpy-ml/blob/master/numpy_ml/neural_nets/layers/layers.py}{numpy-ml}
    中也有navie的backward()实现。

    在Naive实现反向传播时:
    \begin{itemize}
        \item 求$\delta^{l-1} = \dfrac{\partial L}{\partial Z^{l-1}}$时,注意
        $\delta^{l}$需要根据stride在对应位置补零从而还原成stride=1时的大小(即$\delta^{l}$周围
        补一圈零);第$l$层卷积核$W^{l}$需要翻转180度;然后$W^{l}$在还原后的$\delta^{l}$上进行互相关
        操作再按照上述链接推导出的公式得出结果!
        \item 求$\dfrac{\partial J}{\partial W^{l}}, \dfrac{\partial J}{\partial b}$时,不需要翻转!
    \end{itemize}

    \subsubsection{向量化实现}
    如果在CNN layer的计算中,把前向传播变成矩阵乘法(下文提到的向量化),那么在做反向传播的时候,就不需要去关心误差项
    矩阵补零(padding)或者还原成步长(stride)为1时的矩阵了!直接利用类似全连接层的forward()+backward()最后
    通过逆向量化操作(col2im)完成计算!

\hspace*{\fill} \\
    \textbf{向量化im2col()原理及基本实现参考}:
    \begin{itemize}
        \item \href{https://zhuanlan.zhihu.com/p/63974249}{最清晰的im2col过程及实现}(不带padding)
        \item \href{https://zhuanlan.zhihu.com/p/40951745}{前向使用im2col,反向时的卷积操作也是用了im2col}(不带padding),
        没有利用类似全连接层BP的操作+col2im
        \item \href{https://zhuanlan.zhihu.com/p/92722347}{前向im2col+反向col2im(没有Batch维度)}(主要看\textbf{类似全连接层的算法}过程)
    \end{itemize}

    \href{https://github.com/ddbourgin/numpy-ml/blob/master/numpy_ml/neural_nets/layers/layers.py#L2693-L2963}{numpy-ml}
    的Conv2D层就是利用im2col+col2im等向量化手段实现了类似全连接层的操作,输入维度$=(B,H,W,C)$。

    《深度学习入门:基于Python的理论与实现》中第7章 7.4 实现\href{https://github.com/hguomin/deep-learning-from-scratch/blob/master/common/layers.py}{卷积层和池化层}
    forward()使用了im2col();实现backward()使用了col2im(),但是它的
    \href{https://github.com/hguomin/deep-learning-from-scratch/blob/master/common/util.py}{im2col()和col2im()}
    都是用两层for循环实现的。而且是基于PyTorch的输入格式$(B, C, H, W)$。

    (去掉for循环的im2col()实现见\href{https://zhuanlan.zhihu.com/p/64933417}{卷积算法另一种高效实现,as\_strided详解})

\hspace*{\fill} \\
    该书的实现流程如下:
    假设输入张量为$X=(B,C,H,W)$,卷积核为$W=(C_{out}, C_{in}, FH, FW)$,$C_{out}$是卷积核数目,下文中$C=C_{in}$。

    \textbf{forward()}:
    \begin{itemize}
        \item 输入张量$X=(B,C,H,W) \stackrel{im2col()}{\Longrightarrow} X=(B*H_{out}*W_{out}, FH*FW*C_{in})$
        \item 卷积核$W=(C_{out}, C_{in}, FH, FW) \stackrel{reshape}{\Longrightarrow} W=(C_{out}, C_{in}*FH*FW) \stackrel{transpose}{\Longrightarrow} W=(C_{in}*FH*FW, C_{out})$
        \item 输出矩阵$Y=XW+\boldsymbol{1b}^{T}$,这里的$\boldsymbol{1}=(B*H_{out}*W_{out}, 1)$是全1列向量,$\boldsymbol{b}=(B*H_{out}*W_{out}, 1)$也是列向量,$\boldsymbol{1b}^{T}$表明每个卷积核对应一个$b_{i}$元素!
        \item 所以将矩阵还原回张量$Y=(B*H_{out}*W_{out}, C_{out})\stackrel{reshape}{\Longrightarrow} Y=(B, H_{out}, W_{out}, C_{out}) \stackrel{transpose}{\Longrightarrow}Y=(B,C_{out}, H_{out}, W_{out})$
    \end{itemize}

    \textbf{backward()}:
    \begin{itemize}
        \item 给定误差项张量$dY=(B,C_{out}, H_{out}, W_{out})$,先转为二维矩阵形式以便于利用链式法则直接求出导数,$\stackrel{transpose}{\Longrightarrow} dY=(B, H_{out}, W_{out}, C_{out})\stackrel{reshape}{\Longrightarrow} dY=(B*H_{out}*W_{out}, C_{out})$
        \item 根据标量对矩阵求导的链式法则(见matrix derivative.pdf),则有\\ $dW=X^{T}dY,\quad (X=(B*H_{out}*W_{out}, FH*FW*C_{in}))$;
        再将$dW=(FH*FW*C_{in}, C_{out}) \stackrel{transpose}{\Longrightarrow} dW=(C_{out}, C_{in}*H_{out}*W_{out})\stackrel{reshape}{\Longrightarrow} dW=(C_{out}, C_{in}, H_{out}, W_{out})$
        \item $dX=dY\cdot W^{T}, (W=(C_{in}*FH*FW, C_{out})), \quad dX=(B*H_{out}*W_{out}, FH*FW*C_{in}) \stackrel{col2im()}{\Longrightarrow} dX=(B, C_{in}, H, W)$
        \item $d\boldsymbol{b}=dY^{T}\cdot \boldsymbol{1},\quad d\boldsymbol{b}=(C_{out},1)$,相当于np.sum(dY, axis=0),有两种推导方法:\\
        1. 按照链式法则$\boldsymbol{b}\rightarrow Y \rightarrow J$逐分量的推导,可以假设一个很小的例子开始;
        \\
        2. 按照微分法推导$dJ=tr[(\dfrac{\partial J}{\partial Y})^{T}dY]=tr[((\dfrac{\partial J}{\partial Y})^{T}\cdot \boldsymbol{1})^{T}d\boldsymbol{b}]$
    \end{itemize}
    (np.reshape是按照行的顺序叠加)

    具体实现细节见上面给出对应书本源码的链接。需要注意的是,书上的im2col()和col2im()实现思路与
    “向量化im2col()原理及基本实现参考”给出的3个链接中的思路不相同,但除了书上的方法较完善(带pading)
    以及两者针对输入张量的维度顺序不同之外,最终的功能是一样的!

    \section{Attention机制}
    参考\href{https://zhuanlan.zhihu.com/p/106662375}{Attention Review},
    再看\href{https://www.zhihu.com/question/68482809}{JayLou娄杰的回答}!
    能比较系统地了解传统attention(soft)和self-attention的计算过程!

    \subsection{Q,K,V三元组}
    soft-attention中有两种模式:
    \begin{enumerate}
        \item Key=Value=X
        \item Key!=Value
    \end{enumerate}
    假设输入$X_{m\times n}$是由样本行向量拼成的矩阵,$Q_{m\times n},K_{m\times n},V_{m\times n}$,
    且对应的$q_{i},k_{i},v_{i}$均是行向量!(写成列向量也行,后面的推导做转置即可)

    则进行attention得到的特征变化结果:
    \begin{itemize}
        \item $att(\boldsymbol{q_{i}},(K,V))=\sum_{j=1}^{m}\alpha_{j}\boldsymbol{v}_{j}$
        \item $\alpha_{j}=sim(\boldsymbol{q_{i},k_{j}})=softmax(\boldsymbol{q_{i},k_{j}})=\dfrac{\boldsymbol{q_{i}k_{j}}^T}{\sum_{l=1}^{m}\boldsymbol{q_{i}k_{l}}^T}$(这里假设注意力分布采用内积形式)
        \item 所以$att(\boldsymbol{q_{i}},(K,V))=\sum_{j=1}^{m}\dfrac{\boldsymbol{q_{i}k_{j}}^T}{\sum_{l=1}^{m}\boldsymbol{q_{i}k_{l}}^T}\boldsymbol{v}_{j}$
    \end{itemize}
    写成矩阵化形式:
    \begin{itemize}
        \item $att(\boldsymbol{Q,K,V})_{m\times n}=softmax(\boldsymbol{QK}^T,dim=1)\boldsymbol{V}$,此处的softmax作用于矩阵的每一行!
        \item 与前面的求和式进行对比,$(\boldsymbol{QK}^T)_{m\times m}$的第i行经过softmax后中的第j个值对应于$q_{i}$与行向量$k_{j}$的相似度$\alpha_{j}$;
        \item 然后$(\boldsymbol{QK}^T)_{m\times m}$的第i行(1 x m)乘上矩阵$\boldsymbol{V}_{m\times n}$相当于前面求和式中最外层的求和表达式!
        可以用矩阵乘法--->行x矩阵
    \end{itemize}

    \href{http://peterbloem.nl/blog/transformers}{Pytorch实现self-attention \& transformer}


\end{document}